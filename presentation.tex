\documentclass{beamer}
\usetheme[subsectionpage=progressbar]{metropolis}
\usepackage[english]{babel}
\usepackage{ragged2e}
\usepackage{graphicx}
\usepackage{xcolor}
\usepackage{tikz}
\usepackage{tcolorbox}

\setcounter{tocdepth}{2}
\apptocmd{\frame}{}{\justifying}{} % Allow optional arguments after frame.

% Macro to give rounded borders to images.
% https://subscription.packtpub.com/book/hardware-and-creative/9781784395148/4/ch04lvl1sec45
% /cutting-an-image-to-get-rounded-corners
\newsavebox{\picbox}
\newcommand{\cutpic}[3]{
  \savebox{\picbox}{\includegraphics[scale=#2]{#3}}
  \tikz\node [draw, rounded corners=#1, line width=4pt,
  color=white, minimum width=\wd\picbox,
  minimum height=\ht\picbox, path picture={
    \node at (path picture bounding box.center) {
      \usebox{\picbox}};
  }] {};
}

\title{ACM-ICPC Presentation}
\author{Lionel \textsc{Kitihoun}}
\date{}

\begin{document}
\begin{frame}[plain]
    \maketitle
\end{frame}

\begin{frame}{Disclaimer}
Part of the content comes from some slides found on the web. I do not know the authors, so here are the links to the slides.

\begin{itemize}
  \item \url{https://slideplayer.com/slide/1614833}.
  \item \url{https://slideplayer.com/slide/1615102}.
  \item \url{https://slideplayer.com/slide/8333198}.
\end{itemize}
\end{frame}

\begin{frame}{Summary}
  \tableofcontents
\end{frame}

\section{Presentation of the contest}

\begin{frame}{ICPC}
\begin{itemize}
  \justifying
  \item ACM International Collegiate Programming Contest is probably the greatest university-level programming contest in the world.
  \item ACM - Association for Computing Machinery.
  \begin{itemize}
    \justifying
    \item World's largest scientific and educational computing society.
    \item Organizes research conferences.
    \item Publishes journal.
  \end{itemize}
  \item The contest is organized since 1977.
\end{itemize}
\end{frame}

\begin{frame}{ICPC logo}
\begin{figure}
  \centering
  \includegraphics[scale=0.8]{images/icpc-logo.png} \\
  {\tiny\itshape Credit: Wikipedia}
\end{figure}
\end{frame}

\begin{frame}{ACM logo}
\begin{figure}
  \centering
  \includegraphics[scale=0.15]{images/acm-logo.png} \\
  {\tiny\itshape Credit: Wikipedia}
\end{figure}
\end{frame}

\begin{frame}{Purpose}
\begin{itemize}
  \justifying
  \item Provide students the opportunity to sharpen problem solving skills.
  \item It is not just about programming.  
\end{itemize}
\end{frame}

\begin{frame}{Format}
Multi-tiered contest.
\begin{itemize}
  \justifying
  \item Local contest, where each university selects its best teams.
  \item Regional contest (semi-finals).
  \item World finals.
\end{itemize}
\end{frame}

\begin{frame}{Our region}
\begin{itemize}
  \justifying
  \item Africa and Arab region.
  \item The regional contest direction is in Egypt.
  \item Regional site for West Africa is in Benin.
  \item Contest direction in Benin, Togo, Niger, and Burkina-Faso is under by Mapcom Group responsability.
\end{itemize}
\end{frame}

\begin{frame}{World Finals}
\begin{itemize}
  \justifying
  \item Each year, a university is given the honor to host the finals.
  \item It is a great privilege.
  \item Reaching the finals is already an accomplishment.
\end{itemize}
\end{frame}

\begin{frame}{Rules}
\begin{itemize}
  \justifying
  \item Each team consists of three (3) students from the same institute or university.
  \item Each team has access to one computer during the contest.
  \item The contest lasts five (5) hours and there are eight (8) or more problems to solve.
\end{itemize}
\end{frame}

\begin{frame}{Restrictions and quotas}
\begin{itemize}
  \justifying
  \item Each student should have less than five (5) years of university education before the contest.
  \item Students who have previously competed in five (5) regional contests or two (2) world finals are not illegible.
  \item A university can have several teams qualified to a regional contest, but there is at most one team per university at the world finals.
  \item Read the rules at \url{https://icpc.global/regionals/rules}.
  \item Allowed languages are C, C++, Java, and Python.
\end{itemize}
\end{frame}

\begin{frame}{Contest process}
\begin{itemize}
  \justifying
  \item There can be several sites for a regional contest, but there is only one for the world finals.
  \item On each site, teams are gathered in a room.
  \item There is a software that manages the contest.
  \begin{itemize}
    \justifying
    \item It allows you to submit your solutions.
    \item You get back the results of your submissions.
    \item You can ask questions to the judges if you think something is unclear about a problem.
  \end{itemize}
  \item There is a scoreboard.
  \item Each time a team solves a problem, it gets a balloon.
\end{itemize}
\end{frame}

\begin{frame}[c]
\frametitle{A team with its balloons}
\begin{figure}
  \centering
  \cutpic{0.3cm}{0.25}{images/team-fat-bandits.jpg} \\
  {\tiny\itshape Credit: Facebook}
\end{figure}
\end{frame}

\begin{frame}[c]
\frametitle{A contest room}
\begin{figure}
  \centering
  \cutpic{0.3cm}{0.35}{images/itmo-2017.jpg} \\
  {\tiny\itshape Credit: news.itmo.ru}
\end{figure}
\end{frame}

\begin{frame}[c]
\frametitle{World Finals - Porto 2019}
\begin{figure}
  \centering
  \cutpic{0.3cm}{0.35}{images/porto-2019-2.jpg} \\
  {\tiny\itshape Credit: icpc2019.up.pt}
\end{figure}
\end{frame}

\begin{frame}{Ranking}
\begin{itemize}
  \justifying
  \item First, teams are ranked based on the number of solved problems.
  \item When two ore more teams have the same number of solved problems, they are further ranked by time penalty.
  \begin{itemize}
    \justifying
    \item For each solved problem, the number of minutes from the beginning of the contest until the correct solution was submitted.
    \item For each solved problem, twenty (20) minutes for each incorrect submission before the correct solution.
  \end{itemize}
\end{itemize}
\end{frame}

\begin{frame}{Possible answers to submissions}
\begin{itemize}
  \justifying
  \item \textcolor{green}{\textbf{Accepted}}.
  \item \textcolor{red}{\textbf{Wrong Answer} (WA)}. Your program does not give the correct answer.
  \item \textcolor{red}{\textbf{Time Limit Exceeded} (TLE)}. Your program takes too much time. That does not mean it gives the correct answer.
  \item \textcolor{red}{\textbf{Memory Limit Exceeded}}. Your program uses too much memory. Same remark as above.
  \item \textcolor{red}{\textbf{Runtime Error}}.
  \item \textcolor{red}{\textbf{Compilation Error}}. No time penalty for this type of errors.
\end{itemize}
\end{frame}

\begin{frame}{Strategy}
Here are some tips.
\begin{itemize}
  \justifying
  \item Good teamwork is essential.
    \begin{itemize}
      \item You must be complementary and know how to work together.
      \item Each member should have one or several specialties.
      \item Ideally, everyone should have a good programming level.
      \item There is only one computer available, so it must shared.
    \end{itemize}
  \item Once you think you have a solution to a problem, don't rush to the computer. Write solution outline one a paper. It helps a lot.
  \item Identify easy problems and solve them first.
  \item Learn to relax and don't explode under pressure.
  \item Enjoy the moment.
\end{itemize}
\end{frame}

\begin{frame}{Problem categories}
All areas of mathematics.
\begin{itemize}
  \justifying
  \item Geometry.
  \item Graph Theory.
  \item String processing.
  \item Calculus.
  \item Optimization.
  \item etc.
\end{itemize}
\end{frame}

\section{Preparation}

\subsection{Registration}

\begin{frame}{Registration}
In order to compete, students and coach should be registered on ICPC website.

Once your account is created, you must complete your profile.
\begin{itemize}
  \justifying
  \item Name, birthdate, sex, etc.
  \item Academic information.
    \begin{itemize}
      \justifying
      \item Your university or institute.
      \item Post-secondary studies start date.
      \item Branch.
      \item Expected date of graduation.
    \end{itemize}
\end{itemize}
\end{frame}

\begin{frame}{Qualification}
With your account, you can register for a contest.

\begin{itemize}
  \justifying
  \item One of the team member or the coach creates the team and add members.
  \item The contest organizers validate each team. 
  \item If any of the team member did not complete its profile or does not fulfill the requirements, the team will be rejected.
\end{itemize}
\end{frame}

\subsection{Training}

\begin{frame}{An endless, beautiful journey}
\begin{itemize}
  \justifying
  \item You can't be an ace right from the start.
  \item You will need a lot of training to reach a decent level.
  \item It never ends. There will be always something to learn.
  \item It is not about winning the contest, it's all about improving and becoming a good problem solver.
  \item I am not saying that it is not cool to win the contest.
\end{itemize}
\end{frame}

\begin{frame}{The must important thing}
You must have a strong will to learn and get better. It will fuel you through your journey.
\vspace{12pt}
\begin{tcolorbox}[boxrule=0mm]
  \begin{quote}
    Success Is Going from Failure to Failure Without Losing Your Enthusiam.
    \begin{flushright}
      \tiny{---William Chruchill}
    \end{flushright}
  \end{quote}
\end{tcolorbox}
\end{frame}

\begin{frame}{Basic prerequisites}
  \begin{itemize}
    \justifying
    \item English.
    \item You must know how to use a computer.
    \item The computers used during the contest run under Linux, so it's better to get used to it.
    \item You must know at least one of the languages of the contest.
    \item You must know how to compile and run programs from the command-line.
    \item You must also know how I/O redirection works.
  \end{itemize}
\end{frame}

\begin{frame}{More prerequisites}
\begin{itemize}
  \justifying
  \item DSA course.
  \begin{itemize}
    \justifying
    \item You must know common data structures and algorithms.
    \item You must be familiar with the notion of algorithm complexity.
  \end{itemize}
  \item Good level in C++, Java, or Python.
  \begin{itemize}
     \justifying
    \item Object-oriented programming (optional, but useful).
    \item You must know your preferred language strengths and weaknesses.
    \item And its standard library.
  \end{itemize}
  \item Fast typewriting is an advantage. 
\end{itemize}
\end{frame}

\begin{frame}{Learning path}
\begin{itemize}
  \justifying
  \item Follow a competitive programming course.
  \begin{itemize}
    \justifying
    \item Usually, universities have a dedicated course and competitive programming club.
    \item Some websites offer training material (cp-algorithms.com, Geeks for Geeks).
  \end{itemize}
  \item Read books and tutorial on recurrent topics. There is a lot of material available on the web.
  \item Practice and learn on competitive programming websites.
  \begin{itemize}
    \justifying
    \item Codeforces.
    \item Kattis.
    \item Hackerrank.
  \end{itemize}
  \item Participate in contests.
  \begin{itemize}
    \justifying
    \item ACM Contests.
    \item Google Codejam, Kickstart, Hashcode.
    \item Facebook Hacker Cup.
  \end{itemize}
  \item Practice and learn, over and over again.
\end{itemize}
\end{frame}

\begin{frame}{How will we work?}
\begin{itemize}
  \justifying
  \item We will use Kattis (\url{https://open.kattis.com}).
  \begin{itemize}
    \justifying
    \item Create an account if you don't have one.
    \item Read the documentation for your language in the help section.
  \end{itemize}
  \item T-414-ÁFLV: A Competitive Programming Course, by Bjarki Ágúst Guðmundsson. Available at \url{https://algo.is}.
  \item We have a WhatsApp group and a Slack workspace.
\end{itemize}
\end{frame}

\begin{frame}{Course topics}
  \begin{itemize}
    \justifying
    \item Data structures.
    \item Problem solving paradigms.
    \item Greedy algorithms.
    \item Dynamic programming.
    \item Graphs.
    \item Mathematics.
    \item Strings.
    \item Geometry.
  \end{itemize}
\end{frame}


\begin{frame}{Problem set}
For each chapter, there are five problems to solve and two bonus problems.
\end{frame}

\section{Questions}

\begin{frame}[plain]
\begin{center}
  \cutpic{0.1}{0.3}{images/questions.jpg}
\end{center}
\end{frame}

%\begin{frame}{}
%  \begin{itemize}
%    \justifying
%    \item 
%  \end{itemize}
%\end{frame}
\end{document}

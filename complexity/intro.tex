\documentclass{beamer}
\usepackage[english]{babel}
\usepackage{amsmath}
\usepackage{ragged2e}
\usepackage{graphicx}
\usepackage{xcolor}
\usepackage{tikz}
\usepackage{tcolorbox}

\setcounter{tocdepth}{2}
\apptocmd{\frame}{}{\justifying}{} % Allow optional arguments after frame.

% Macro to give rounded borders to images.
% https://subscription.packtpub.com/book/hardware-and-creative/9781784395148/4/ch04lvl1sec45
% /cutting-an-image-to-get-rounded-corners
\newsavebox{\picbox}
\newcommand{\cutpic}[3]{
  \savebox{\picbox}{\includegraphics[scale=#2]{#3}}
  \tikz\node [draw, rounded corners=#1, line width=4pt,
  color=white, minimum width=\wd\picbox,
  minimum height=\ht\picbox, path picture={
    \node at (path picture bounding box.center) {
      \usebox{\picbox}};
  }] {};
}

\title{Algorithm Complexity}
\author{Lionel \textsc{Kitihoun}}
\date{}

\begin{document}
\begin{frame}[plain]
\maketitle
\end{frame}

\begin{frame}{Introduction}
\justifying
We often need to evaluate a program cost, in terms of:
\begin{itemize}
  \item processor usage (number of operations needed to solve the problem),
  \item memory usage (how much RAM will we need).
\end{itemize}
\end{frame}

\begin{frame}{Why?}
\justifying
\begin{itemize}
  \item We often have limited resources.
    \begin{itemize}
      \item We have a lot of other things to do.
      \item We need the result of our program as soon as possible.
      \item Computation power and memory are costly.
    \end{itemize}
  \item We have several programs that can solve our problem.
    \begin{itemize}
      \item We need to pick the best one.
    \end{itemize}
\end{itemize}
\end{frame}

\begin{frame}{Algorithm complexity}
\justifying
Complexity of an algorithm is a measure of the amount of time or space required by an algorithm for an input of a given size (n)\footnote{\url{http://www.dcs.gla.ac.uk/~pat/52233/complexity.html}}.
\end{frame}

\begin{frame}{Your task}
\justifying
You should be able to deduce the complexity of a given algorithm in order to decide if it fits your need.
\end{frame}

\begin{frame}{Initiation examples}
\justifying
Three examples.
\begin{itemize}
  \item Sum of the first $N$ positive integers.
  \item Primality test.
  \item Fast exponentiation.
\end{itemize}
\end{frame}

\begin{frame}[fragile]
\justifying
\frametitle{Sum of the first N positive integers (Naive)}
\begin{verbatim}
  function sum(n: int): int
  var
    s: integer
    i: integer
  begin
    s := 0
    for i := 1 to n
        s = s + i
    return s
  end
\end{verbatim}
\end{frame}

\begin{frame}{Sum of the first N positive integers (Naive)}
\justifying
So, how many operations are needed?
\end{frame}

\begin{frame}{Answer}
\justifying
This program needs $N + 1$ operations to get the answer.
\end{frame}

\begin{frame}[fragile]
\frametitle{Sum of the first $N$ positive integers (Better)}
\begin{verbatim}
  function sum(n: int): int
  var
    s: integer
  begin
    s := n * (n + 1) / 2
    return s
  end
\end{verbatim}
\end{frame}

\begin{frame}{Number of operations}
\justifying
This version only needs one instruction\footnote{Okay, three operations. But you get the idea.}.
\end{frame}

\begin{frame}{Make your choice}
I prefer the second version.
\begin{itemize}
  \item It is faster, imagine if $N = 10^{10}$.
  \item It makes me look smart.
\end{itemize}
\end{frame}

\begin{frame}{Second example: primality test}
\justifying
Given a number, check if it is prime or not.
\end{frame}

\begin{frame}{Primality test: first take}
Easy: just divide the number by all other integers below it.
\end{frame}

\begin{frame}[fragile]
\frametitle{Primality test pseudo-code}
\begin{verbatim}
  function is_prime(n: int): boolean
    var
      i: integer
    begin
      for i := 2 to n-1
        if n mod i = 0
          return false
      return true
  end
\end{verbatim}
\end{frame}

\begin{frame}{Question}
  \justifying
  How many operations are needed?
\end{frame}

\begin{frame}{Answer}
\justifying
\begin{itemize}
  \item We loop $N-2$ times.
  \item In each loop we do one test.
  \item $N-2$ operations overall.
\end{itemize}
\end{frame}

\begin{frame}[fragile]
\frametitle{Primality test: better version}
\begin{verbatim}
function is_prime(n: int): boolean
var
  i: integer
begin
  while i * i <= n
    if n mod i = 0
      return false
  return true
\end{verbatim}
\end{frame}

\begin{frame}{Cost}
\justifying
\begin{itemize}
  \item We check every number up to $\sqrt{N}$.
  \item So, we do at most $\sqrt{N}$ operations.
  \item Better than the other version.
\end{itemize}
\end{frame}

\begin{frame}{Last example: fast exponentiation}
Given a real number $x$ and an integer $n$, compute $x^n$.
\end{frame}

\begin{frame}{Fast exponentiation insight}
\justifying
\begin{itemize}
  \item No loop needed.
  \item We just need one key observation.
\end{itemize}
\end{frame}

\begin{frame}{Fast exponentiation trick}
\[
  x^n= 
  \begin{cases}
    x^{n / 2} * x^{n / 2},\	if\ n\ is\ even,\\
    x * x^{n / 2} * x^{n / 2},\ if\ n\ is\ odd
  \end{cases}
\]
\end{frame}

\begin{frame}[fragile]
\frametitle{Pseudo-code}
\begin{verbatim}
function exp(x: real, n: int): real
var
  h: integer
  p: real
begin
  if n = 0
    return 1
  h = n / 2
  p = exp(x, h)
  if n mod 2 = 0
    return h * h
  else
    return x * h * h
end
\end{verbatim}
\end{frame}

\begin{frame}{Cost}
\justifying
\begin{itemize}
  \item You need a bit of analysis.
  \item The number of recursive call is roughly equal to $ln_{2}(n)$.
  \item We will say that we need $ln(n)$ recursive calls.
\end{itemize}
\end{frame}

\begin{frame}{Next}
  This is just an introduction. Now, the real slides.
\end{frame}
\end{document}
